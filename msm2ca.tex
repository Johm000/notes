\documentclass[12pt]{article}
\usepackage{amsmath}
\usepackage{amsthm}

% Font stuff
\usepackage[T1]{fontenc}
\usepackage[sfmath]{kpfonts}
\renewcommand*\familydefault{\sfdefault}

\usepackage{fullpage}

\newtheorem*{defn}{Definition}
\newtheorem*{lemma}{Lemma}
\newtheorem*{prop}{Proposition}
\newtheorem*{crl}{Corollary}
\newtheorem*{thm}{Theorem}

\date{\today}

\input{./include/keywords.tex}

% SET THIS STUFF - info for front page.
\newcommand{\nameOfTheModule}{MSM2Ca - Linear Algebra}
\newcommand{\nameOfTheAuthor}{Will Ridgers}
\newcommand{\nameOfTheLecturer}{Dr Sergey Shpectorov}

\begin{document}
\newcommand{\HRule}{\rule{\linewidth}{0.5mm}}

\begin{titlepage}
\begin{center}


% Upper part of the page
\textsc{\LARGE University of Birmingham}\\[1.5cm]

\textsc{\Large Summary of lectures}\\[0.5cm]


% Title
\HRule \\[0.4cm]
{ \huge \bfseries \nameOfTheModule}\\[0.4cm]
\HRule \\[1.5cm]

% Author
\nameOfTheAuthor

\vfill

% Bottom of the page
{\large \today}

\end{center}
\end{titlepage}
\tableofcontents
\pagebreak

\section{Introduction}
	\subsection{Fields}
	The set of scalars will often be $\mathbb{R}$, $\mathbb{C}$, or $\mathbb{Q}$. These sets are mathematical structures called fields. Other examples of fields would be
	\begin{itemize}
		\item $\mathbb{Z}_p$ (integers of modulo $p$) where $p$ is a prime integer.
		\item $\mathbb{F}_q$ where $q = p^q \not{=} 1$ is a prime power. This is a unique finite field.
	\end{itemize}
	
	\subsection{Working with finite fields}
	In $\mathbb{F}_p = \mathbb{Z}_p$ you can simply do the operations in modulo $p$.
	\[
		x \oplus y := x + y mod p,
	\]
	\[
		x \otimes y := xy mod p
	\]
	Subtration and division are subsituted with addition and multiplication:
	\[
		x - y := x + (-y),
	\]
	\[
		\frac{x}{y} := xy^{-1}
	\]
	
	\subsubsection{Operations for $\mathbb{F}_4$}
	For any finite field you can use precomupted addition/multiplication tables.
	\[
		\begin{array}{c | c c c c}
			+ & 0 & 1 & a & b \\
			\hline
			0 & 0 & 1 & a & b \\
			1 & 1 & 0 & b & a \\ 
			a & a & b & 0 & 1 \\
			b & b & a & 1 & 0
		\end{array}
		\quad
		\begin{array}{c|cccc}
			\cdot & 0 & 1 & a & b \\
			\hline
			0 & 0 & 0 & 0 & 0 \\
			1 & 0 & 1 & a & b \\
			a & 0 & a & b & 1 \\
			b & 0 & b & 1 & a
		\end{array}
	\]
	In $\mathbb{F}_4$, note that $-0 = 0$, $-1 = 1$, $-a = a$, and $-b = b$.

\section{Vector Spaces}
	\begin{defn}
		A vector space over the field of scalars $\mathbb{F}_4$ is a set $V$, whose elements are vectors, together with two binary operations addition and multiplication. The following axioms hold for addition
		\begin{itemize}
			\item Associative: $\forall u,v,w \in V, \quad u + ( v + w ) = (u + v) + w$
			\item Existance of zero element: $\forall u \in V, \quad \exists 0 \in V$ such that $u + 0 = u$
			\item Existance of inverse element: $\forall u \in V, \quad \exists (-u) \in V$ such that $u + (-u) = 0$
			\item Commutative: $\forall u,v \in V, \quad u + v = v + u$
		\end{itemize}
		And the axioms for multiplication are
		\begin{itemize}
			\item Associative: $\forall a,b \in \mathbb{F}_4, v \in V, \quad a(bv) = (ab)v$.
			\item Distributive: $\forall a,b \in \mathbb{F}_4, v,u \in V$ then
					\[
						a(u+v) = au + av,
					\]
					\[
						(a+b)v = av + bv
					\]
			\item Existance of identity element: $1v = v \forall v \in V$
		\end{itemize}
	\end{defn}
	
	\subsection{Basic properties}
	\begin{thm}
		Suppose that $V$ is a vector space over a field of scalars $\mathbb{F}$. Then the following axioms hold:
		\begin{itemize}
			\item $\forall u,v,w \in V$ if $u + v = u + w$ then $v=w$ (cancellation of sums)
			\item $0v = 0, \quad \forall v \in V$
			\item $(-1)v = -v, \quad \forall v \in V$
			\item $a0 = 0, \quad \forall a \in \mathbb{F}$
		\end{itemize}
	\end{thm}
	
	\begin{thm}
		Suppose $V$ is a vector space over $\mathbb{F}$, and $a \in \mathbb{F}, \quad u,v \in V$
		\begin{itemize}
			\item If $av = 0$ then either $a = 0$ or $v = 0$
			\item $a \not{=} 0$, if $au = av$ then $u=v$
			\item $u \not{=} 0$, if $au = bu$ then $a=b$
		\end{itemize}
	\end{thm}
	
	\subsection{Set of functions}
	For two sets $X$ and $Y$, the set of functions (mappings) from $X$ to $Y$ is denoted by $Y^X$. If $X$ and $Y$ are finite sets with
	$n$ and $m$ elements, then size of $Y^X$, is exactly $m^n$.
	
\section{Subspaces}

	\subsection{Vector subspaces}
	\begin{defn}
		Let $V$ be a vector space defined over the field of scalars $\mathbb{F}$ and $U \subset V$, $U$ is subspace if it is a vector space over $\mathbb{F}$ for the operations inherited from $V$. 
		To be a subspace $U$ must
		\begin{itemize}
			\item Contain the zero vector 0
			\item Closed with respect to addition
			\item Closed with respect to multiplication
			\item Closed with respect to taking negative vectors
		\end{itemize}
	\end{defn}
	
	\begin{thm}
		A nonempty subset $U$ of a vector space $V$ is a subspace if and only if $U$ is closed with respect to addition and scalar multiplication.
	\end{thm}
	
	\subsection{Intersection of subspaces}
	\begin{thm}
		If $\{ U_i | i \in I\}$ is a collection of subspaces of a vector space $V$ then the intersection of $W := \cup_{i \in I} U_i$ is again a subspace of $V$.
	\end{thm}
	
	\subsection{Subspace spanned by a set of vectors}
	\begin{thm}
		Suppose $X$ is a set of vectors from a vector space $V$. The subspace spanned by $X$ is the smallest subspace $W$ of $V$ containing $X$. We write that $W = \left< X \right>$.
	\end{thm}
	
	\subsection{Linear combinations}
	\begin{defn}
		
	\end{defn}

\section{Bases and dimension}
\section{Linear mappings}
\section{Isomorphisms}
\section{Matrix of linear mapping}
\section{Linear transformations}
\section{Eigenvectors}



\end{document}