\documentclass[12pt]{article}
\usepackage{amsmath}
\usepackage{amsthm}

% Font stuff
\usepackage[T1]{fontenc}
\usepackage[sfmath]{kpfonts}
\renewcommand*\familydefault{\sfdefault}

\usepackage{fullpage}

\theoremstyle{definition}
\newtheorem*{defn}{Definition}
\newtheorem*{exmp}{Example}


\theoremstyle{plain}
\newtheorem*{thm}{Theorem}
\newtheorem*{lemma}{Lemma}
\newtheorem*{prop}{Proposition}
\newtheorem*{crl}{Corollary}

\theoremstyle{remark}
\newtheorem*{nb}{Note}
\newtheorem*{remark}{Remark}
\newtheorem*{fact}{Fact}

\date{\today}

\input{./include/keywords.tex}

% SET THIS STUFF - info for front page.
\newcommand{\nameOfTheModule}{MSM2Ca - Linear Algebra}
\newcommand{\nameOfTheAuthor}{Will Ridgers}
\newcommand{\nameOfTheLecturer}{Dr Sergey Shpectorov}

\begin{document}
\newcommand{\HRule}{\rule{\linewidth}{0.5mm}}

\begin{titlepage}
\begin{center}


% Upper part of the page
\textsc{\LARGE University of Birmingham}\\[1.5cm]

\textsc{\Large Summary of lectures}\\[0.5cm]


% Title
\HRule \\[0.4cm]
{ \huge \bfseries \nameOfTheModule}\\[0.4cm]
\HRule \\[1.5cm]

% Author
\nameOfTheAuthor

\vfill

% Bottom of the page
{\large \today}

\end{center}
\end{titlepage}
\tableofcontents
\pagebreak

\section{Introduction}
	\subsection{Fields}
	The set of scalars will often be $\mathbb{R}$, $\mathbb{C}$, or $\mathbb{Q}$. These sets are mathematical structures called fields. Other examples of fields would be
	\begin{itemize}
		\item $\mathbb{Z}_p$ (integers of modulo $p$) where $p$ is a prime integer.
		\item $\mathbb{F}_q$ where $q = p^q \not{=} 1$ is a prime power. This is a unique finite field.
	\end{itemize}
	
	\subsection{Working with finite fields}
	In $\mathbb{F}_p = \mathbb{Z}_p$ you can simply do the operations in modulo $p$.
	\[
		x \oplus y := x + y mod p,
	\]
	\[
		x \otimes y := xy mod p
	\]
	Subtration and division are subsituted with addition and multiplication:
	\[
		x - y := x + (-y),
	\]
	\[
		\frac{x}{y} := xy^{-1}
	\]
	
	\subsubsection{Operations for $\mathbb{F}_4$}
	For any finite field you can use precomupted addition/multiplication tables.
	\[
		\begin{array}{c | c c c c}
			+ & 0 & 1 & a & b \\
			\hline
			0 & 0 & 1 & a & b \\
			1 & 1 & 0 & b & a \\ 
			a & a & b & 0 & 1 \\
			b & b & a & 1 & 0
		\end{array}
		\quad
		\begin{array}{c|cccc}
			\cdot & 0 & 1 & a & b \\
			\hline
			0 & 0 & 0 & 0 & 0 \\
			1 & 0 & 1 & a & b \\
			a & 0 & a & b & 1 \\
			b & 0 & b & 1 & a
		\end{array}
	\]
	In $\mathbb{F}_4$, note that $-0 = 0$, $-1 = 1$, $-a = a$, and $-b = b$.

\section{Vector Spaces}
	\begin{defn}
		A vector space over the field of scalars $\mathbb{F}_4$ is a set $V$, whose elements are vectors, together with two binary operations addition and multiplication. The following axioms hold for addition
		\begin{itemize}
			\item Associative: $\forall a,b,c \in V, \quad a + ( b + c ) = (a + b) + c$.
			\item Existance of zero element: $\forall a \in V, \quad \exists 0 \in V$ such that $a + 0 = a$.
			\item Existance of inverse element: $\forall a \in V, \quad \exists (-a) \in V$ such that $a + (-a) = 0$.
			\item Commutative: $\forall a,b \in V, \quad a + b = b + a$.
		\end{itemize}
		And the axioms for multiplication are
		\begin{itemize}
			\item Associative: $\forall a,b \in \mathbb{F}_4, \quad a(bc) = (ab)c$.
			\item Distributive: $\forall a,b,c \in \mathbb{F}_4$ then
			\item Existance of inverse element:
		\end{itemize}
	\end{defn}
	
	
\section{Subspaces}
\section{Bases and dimension}
\section{Linear mappings}
\section{Isomorphisms}
\section{Matrix of linear mapping}
\section{Linear transformations}
\section{Eigenvectors}



\end{document}