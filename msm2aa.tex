\documentclass[12pt]{article}
\usepackage{amsmath}
\usepackage{amsthm}

% Font stuff
\usepackage[T1]{fontenc}
\usepackage[sfmath]{kpfonts}
\renewcommand*\familydefault{\sfdefault}

\usepackage{fullpage}

\theoremstyle{definition}
\newtheorem*{defn}{Definition}
\newtheorem*{exmp}{Example}


\theoremstyle{plain}
\newtheorem*{thm}{Theorem}
\newtheorem*{lemma}{Lemma}
\newtheorem*{prop}{Proposition}
\newtheorem*{crl}{Corollary}

\theoremstyle{remark}
\newtheorem*{nb}{Note}
\newtheorem*{remark}{Remark}
\newtheorem*{fact}{Fact}

\date{\today}


% SET THIS STUFF - info for front page.
\newcommand{\nameOfTheModule}{MSM2template}
\newcommand{\nameOfTheAuthor}{Will Ridgers}
\newcommand{\nameOfTheLecturer}{Dr Qianxi Wang}

\begin{document}
\newcommand{\HRule}{\rule{\linewidth}{0.5mm}}

\begin{titlepage}
\begin{center}


% Upper part of the page
\textsc{\LARGE University of Birmingham}\\[1.5cm]

\textsc{\Large Summary of lectures}\\[0.5cm]


% Title
\HRule \\[0.4cm]
{ \huge \bfseries \nameOfTheModule}\\[0.4cm]
\HRule \\[1.5cm]

% Author
\nameOfTheAuthor

\vfill

% Bottom of the page
{\large \today}

\end{center}
\end{titlepage}
\tableofcontents
\pagebreak

\section{Functions of Several Real Variables}
	\begin{defn} 
		For a function $z = f(x,y)$ a vertical section if the graph of $z = f(x,c)$ or $z = f(c,y)$ 
		for some constant $c$. A level curve is the curve $c = f(x,y)$.
	\end{defn}

\section{Partial derivatives and chain rule}

	\begin{defn}
		The function $f: \mathbb{R}^3 \to \mathbb{R}$ is said to have a partial derivative with respect to $x$
		at the point $(x_0, y_0, z_0)$ if there exists a limit
			\[
				\frac{\partial f (x_0, y_0, z_0)}{\partial x} \equiv \lim_{\Delta x \to 0} 
				\frac{f(x_0 + \Delta x, y_0, z_0) - f(x_0, y_0, z_0)}{\Delta x}
			\]
		which is called the partial derivative of $f$ with respect to $x$.
	\end{defn}
	
	Derivatives with respect to $y$ and $z$ are defined in a similar way.

	\begin{defn}
		The function $f(x,y,z)$ is called differentiable at $(x_0, y_0, z_0)$ if 
		$\Delta f = f(x,y,z) - f(x_0, y_0, z_0)$ can be expressed as
		\[
			\Delta f = f_x(x_0, y_0, z_0) \Delta x + f_y(x_0, y_0, z_0) \Delta y + f_z(x_0, y_0, z_0) \Delta z + o(\rho)
		\]
		where $\Delta x = x - x_0$, $\Delta y = y - y_0$, $\Delta z = z - z_0$ and 
		\[
			\rho = \sqrt{ (\Delta x)^2 + (\Delta y)^2 + (\Delta z)^2 }
		\]
	\end{defn}
	The notation $o(\rho)$ means going to $0$ faster than $\rho$. Note, this is infinitely small so can be ignored.




\section{Vector and vector geometry}
\section{Gradient vector and directional derivative}

\end{document}