\documentclass[12pt]{article}
\usepackage{amsmath}
\usepackage{amsthm}

% Font stuff
\usepackage[T1]{fontenc}
\usepackage[sfmath]{kpfonts}
\renewcommand*\familydefault{\sfdefault}

\usepackage{fullpage}

\theoremstyle{definition}
\newtheorem*{defn}{Definition}
\newtheorem*{exmp}{Example}


\theoremstyle{plain}
\newtheorem*{thm}{Theorem}
\newtheorem*{lemma}{Lemma}
\newtheorem*{prop}{Proposition}
\newtheorem*{crl}{Corollary}

\theoremstyle{remark}
\newtheorem*{nb}{Note}
\newtheorem*{remark}{Remark}
\newtheorem*{fact}{Fact}

\date{\today}

\input{./include/keywords.tex}

% SET THIS STUFF - info for front page.
\newcommand{\nameOfTheModule}{MSM2Aa}
\newcommand{\nameOfTheAuthor}{Will Ridgers}
\newcommand{\nameOfTheLecturer}{Dr Qianxi Wang}

% TODO:
% - check methods are concise as possible
% - add gfx
%
%
%

\begin{document}
\newcommand{\HRule}{\rule{\linewidth}{0.5mm}}

\begin{titlepage}
\begin{center}


% Upper part of the page
\textsc{\LARGE University of Birmingham}\\[1.5cm]

\textsc{\Large Summary of lectures}\\[0.5cm]


% Title
\HRule \\[0.4cm]
{ \huge \bfseries \nameOfTheModule}\\[0.4cm]
\HRule \\[1.5cm]

% Author
\nameOfTheAuthor

\vfill

% Bottom of the page
{\large \today}

\end{center}
\end{titlepage}
\tableofcontents
\pagebreak

\section{Functions of Several Real Variables}
	\subsection{Functions of Several Real Variables}
		\begin{defn} 
			For a function $z = f(x,y)$ a vertical section if the graph of $z = f(x,c)$ or $z = f(c,y)$ 
			for some constant $c$. A level curve is the curve $c = f(x,y)$.
		\end{defn}

	\subsection{Partial derivatives and chain rule}

		\subsubsection{Partial derivatives}
		\begin{defn}
			The function $f: \mathbb{R}^3 \to \mathbb{R}$ is said to have a partial derivative with respect to $x$
			at the point $(x_0, y_0, z_0)$ if there exists a limit
				\[
					\frac{\partial f (x_0, y_0, z_0)}{\partial x} \equiv \lim_{\Delta x \to 0} 
					\frac{f(x_0 + \Delta x, y_0, z_0) - f(x_0, y_0, z_0)}{\Delta x}
				\]
			which is called the partial derivative of $f$ with respect to $x$.
		\end{defn}
		
		Derivatives with respect to $y$ and $z$ are defined in a similar way.

		\subsubsection{Differentiable functions}
		\begin{defn}
			The function $f(x,y,z)$ is called differentiable at $(x_0, y_0, z_0)$ if 
			$\Delta f = f(x,y,z) - f(x_0, y_0, z_0)$ can be expressed as
			\[
				\Delta f = f_x(x_0, y_0, z_0) \Delta x + f_y(x_0, y_0, z_0) \Delta y + f_z(x_0, y_0, z_0) \Delta z + o(\rho)
			\]
			where $\Delta x = x - x_0$, $\Delta y = y - y_0$, $\Delta z = z - z_0$ and 
			\[
				\rho = \sqrt{ (\Delta x)^2 + (\Delta y)^2 + (\Delta z)^2 }
			\]
		\end{defn}
		The notation $o(\rho)$ means going to $0$ faster than $\rho$. Note, this is infinitely small so can be ignored.
		
		\subsubsection{Sufficiently smooth}
		\begin{defn}
			(Informal) A function is said to be sufficiently smooth if the function and as many of it's partial derivatives
			as required are continuous where they need to be.
		\end{defn}
		
		\subsubsection{Chain rule}
		\begin{defn}
			Chain Rule for two variables. Let $f$ be a function of the variables $x_1, x_2, \cdots , x_n$, $f = f(x_1, x_2, \cdots , x_n)$ 
			where each $x_j$ is a function of the variables $t_1, t_2, \cdots , t_n$, $x_j = x_j(t_1, t_2, \cdots , t_n)$. Provided
			that $f$ and $x_j$ are sufficiently smooth, then
			\[
				\frac{\partial f}{\partial t_i} = \frac{\partial f}{\partial x_1}\frac{\partial x_1}{\partial t_i} + 
												\frac{\partial f}{\partial x_2}\frac{\partial x_2}{\partial t_i} + 
												\cdots +
												\frac{\partial f}{\partial x_n}\frac{\partial x_n}{\partial t_i}
			\]
		\end{defn}


	\subsection{Vector and vector geometry}

		\subsubsection{Vectors}
		\begin{defn}
			A vector has both a magnitude and a direction. Vectors may be denoted as $\mathbf{u}$, $\vec{u}$, or $\underline{u}$.
			The magnitude of a vector is denoted by 
			\[
				|\mathbf{u}| = \sqrt{u_1^2 + u_2^2 + u_3^2}
			\]
			A unit vector has length $1$ and in the Cartesian coordinate system $\mathbf{i}$, $\mathbf{j}$, and $\mathbf{k}$ denote the the unit 
			coordinate vectors along x, y, z-axes respectively.
		\end{defn}
		
		\subsubsection{Scalar (dot) product}
		\begin{defn}
			Given two vectors $\mathbf{a} = a_1 \mathbf{i} + a_2 \mathbf{j} + a_3 \mathbf{k}$,
			$\mathbf{b} = b_1 \mathbf{i} + b_2 \mathbf{j} + b_3 \mathbf{k}$ the scalar (or dot) product of the two is
			\[
				\mathbf{a} \cdot \mathbf{b} = |\mathbf{a}||\mathbf{b}|\cos{\theta} = a_1 b_1 + a_2 b_2 + a_3 b_3
			\]
			where $\theta$ is the angle between $\mathbf{a}$ and $\mathbf{b}$.
		\end{defn}
		
		\subsubsection{Vector (cross) product}
		\begin{defn}
			Given two vectors $\mathbf{a} = a_1 \mathbf{i} + a_2 \mathbf{j} + a_3 \mathbf{k}$,
			$\mathbf{b} = b_1 \mathbf{i} + b_2 \mathbf{j} + b_3 \mathbf{k}$ the vector (or cross) product of the two is
			\[
				\mathbf{a} \times \mathbf{b} = |\mathbf{a}||\mathbf{b}|\sin{\theta}\mathbf{n} =
				\begin{vmatrix}
					\mathbf{i} & \mathbf{j} & \mathbf{k} \\
					a_1 & a_2 & a_3 \\
					b_1 & b_2 & b_3 
				\end{vmatrix} 
			\]
		\end{defn}
		
		\subsubsection{Straight line equation}
		\begin{defn}
			The vector equation of a straight line is
			\[
				\mathbf{r} = \mathbf{a} + t \mathbf{d}
			\]
			for $t \in \mathbb{R}$, where $\mathbf{a}$ is a point on the line, $\mathbf{d}$ is parallel to the line, 
			and $\mathbf{r}$ is the position vector of a general point on the line.
		\end{defn}
		
		\subsubsection{3 dimensional plane equation}
		\begin{defn}
			The vector equation of a plane in 3 dimensions is
			\[
				\mathbf{r} \cdot \mathbf{n} = \mathbf{a} \cdot \mathbf{n}
			\]
			where $\mathbf{a}$ is the position vector of a point in the plane, $\mathbf{n}$ is a vector normal to
			the plane and $\mathbf{r}$ is the position vector of a general point in the plane.
		\end{defn}
		
		\subsubsection{Parametric form}
		A curve $C$ can be expressed using parametric form
		\[
			\mathbf{r}(t) = x(t)\mathbf{i} + y(t)\mathbf{j} + z(t)\mathbf{k}
		\]
		where $a \leq t \leq b$. Thus the initial point $A$ is $(x(a),y(a),z(a))$ and the terminal point $B$ at $(x(b),y(b),z(b))$.
		$\mathbf{r}(t)$ is a mapping on the interval $[a,b]$ on the t-axis to the curbe in xyz-space.
		
		\subsubsection{Tangent vector of a curve}
		For a curve represented in parametric form (see above), for each $t$ there is a point $\mathbf{r}(t)$ on the curve. The tangent
		vector of the curve can be found by
		\begin{align*}
			\frac{d\mathbf{r}}{dt}& = \lim_{\Delta t \to 0} \frac{\mathbf{r}(t+\Delta t) - \mathbf{r}(t)}{\Delta t} \\
							& = \lim_{\Delta t \to 0} \frac{x(t+\Delta t) - x(t)}{\Delta t} \mathbf{i} + 
								  \lim_{\Delta t \to 0} \frac{y(t+\Delta t) - y(t)}{\Delta t} \mathbf{j} + 
								  \lim_{\Delta t \to 0} \frac{z(t+\Delta t) - z(t)}{\Delta t} \mathbf{k} \\
							& = \frac{dx}{dt} \mathbf{i} + \frac{dy}{dt} \mathbf{j} + \frac{dz}{dt} \mathbf{k}
		\end{align*} 
		The unit tangent vector $\tau$ is $\tau \equiv \frac{\mathbf{r}_t}{|\mathbf{r}_t|}$. In general $\mathbf{r}' \not{=} 0$ for
		a sufficiently smooth function.
		
	\subsection{Gradient vector and directional derivative}
		
		\subsubsection{Gradient vector (grad/nabla/del)}
		\begin{defn}
			Let $f:\mathbb{R}^3 \to \mathbb{R}$ be a function. If the partial derfivatives of $f$ with respect to $x$, $y$, and $z$
			at the point $(a,b,c)$ then the gradient vector of $f$ at $(a,b,c)$ is defined to be the vector
			\[
				\nabla f(a,b,c) = \frac{\partial f(a,b,c)}{\partial x}\mathbf{i} + 
									\frac{\partial f(a,b,c)}{\partial y}\mathbf{j} + 
									\frac{\partial f(a,b,c)}{\partial z}\mathbf{k}
			\]
		\end{defn}
		
		\subsubsection{Key points}
		\begin{itemize}
			\item The value of $\nabla f$ at a point is independent of the coordinate system.
			\item $\nabla f$ is a vector function in n dimensions where f has n variables.
			\item $\nabla$ is a linear operator, i.e $\nabla (\lambda f + \mu g) = \lambda \nabla f + \mu \nabla g$.
		\end{itemize}
		
		\subsubsection{Grad as normal}
		\begin{thm}
			Let $f:\mathbb{R}^2 \to \mathbb{R}$ be a sufficiently function, and $(a,b)$ be a point on the level curve $f(x,y)=k$ for some
			constant $k$. If $\nabla f (a,b) \not{=} 0$, then the vector $\nabla f (a,b)$ is normal to the level curbe $f(x,y)=k$ at the point $(a,b)$.
		\end{thm}
		
		\subsubsection{Finding tangent plane to a surface}
		To find the tangent plane to a surface $z=F(x,y)$ or $G(x,y,z)=k$ at the point $\mathbf{a} = (a,b,c)$:
		\begin{enumerate}
			\item Write the surface in the form of the level surface
				\[
					f(x,y,z) = F(x,y) - z = 0
				\]
				or
				\[
					f(x,y,z) = G(x,y,z) - k = 0
				\]
			\item Calculate the normal vector to the surface at the point $\mathbf{a}$
				\[
					\mathbf{n} = \nabla f ( \mathbf{a} )
				\]
			\item The tangent plane is
				\[
					\mathbf{r} \cdot \mathbf{n} = \mathbf{a} \cdot \mathbf{n}
				\]
		\end{enumerate}
		
		\subsubsection{Directional derivatives}
		\begin{defn}
			Suppose that $f:\mathbb{R}^3 \to \mathbb{R}$ and $\mathbf{u} = u_1 \mathbf{i} + u_2 \mathbf{j} + u_3 \mathbf{k}$ is a unit vector in $\mathbb{R}^3$.
			The directional derivative of $f$ at $(x,y,z)$ in the direction of $\mathbf{u}$ is defined to be
			\[
				D_u f(x,y,z) = \lim_{h \to 0^{+}} \frac{f(x+hu_1, y+hu_2, z+hu_3) - f(x,y,z)}{h}
			\]
		\end{defn}
		
		\begin{lemma}
			Let $f:\mathbb{R}^3 \to \mathbb{R}$ be a sufficiently smooth, and $\mathbf{u} = u_1 \mathbf{i} + u_2 \mathbf{j} + u_3 \mathbf{k}$ 
			be a unit vector in $\mathbb{R}^3$. If it exists, then
			\[
				\frac{d}{dt} f(a + tu_1, b + tu_2, c + tu_3) | _ t=0 = D_u f(a,b,c)
			\]
		\end{lemma}
		
		\subsubsection{Relation between gradient and directional derivative}
		\begin{thm}
			Let $f:\mathbb{R}^3 \to \mathbb{R}$ be a sufficiently smooth, and $\mathbf{u} = u_1 \mathbf{i} + u_2 \mathbf{j} + u_3 \mathbf{k}$ 
			be a unit vector in $\mathbb{R}^3$. Then
			\[
				D_u f(a,b,c) = \mathbf{u} \cdot \nabla f(a,b,c)
			\]
		\end{thm}
		
		\begin{itemize}
			\item The directional derivative is a scalar function not a vector.
			\item $D_u$ is linear; $D_u(\lambda f + \mu g) = \lambda D_u f + \mu D_u g$.
		\end{itemize}
		
		\subsubsection{Calculating directional derivatives}
		To calculate the directional derivatibe of a function $f$ in the direction of the vector $\mathbf{u}$:
		\begin{enumerate}
			\item Find the unit vector $\hat{\mathbf{u}} = \frac{\mathbf{u}}{|\mathbf{u}|}$
			\item Find $\nabla f$
			\item Calculate $D_{\hat{\mathbf{u}}} f = \nabla f \cdot \hat{\mathbf{u}}$
		\end{enumerate}
		
		\subsubsection{Steepst ascent theorem}
		\begin{thm}
			Let $f$ be a function of several real variables. Suppose that $\nabla f$ exists and is non-zero at the point $\mathbf{a}$. 
			Then the direction of the vector $\nabla f(\mathbf{a})$ is always the direction of maximum increase of the function $f$ at the point $\mathbf{a}$.
		\end{thm}
		
		\subsubsection{Finding paths of maximum increase}
		To calculate the path of maximum increase of a function $f$ of two variables starting at point $(a,b)$:
		\begin{enumerate}
			\item Let the path be $\mathbf{r}(t) = x(t)\mathbf{i} + y(t)\mathbf{j}$
			\item $\nabla f$ at a point $(x,y)$ is always in the direction of the maximum increase (by Steepst ascent theorem) of the function $f$ at the point,
					so the tangent vector to the path  $\mathbf{r}'(t) = x'(t)\mathbf{i} + y'(t)\mathbf{j}$ is parallel to $\nabla f = f_x \mathbf{i} + f_y \mathbf{j}$
			\item Since the two vectors are parallel we have
					\[
						x'(t) \mathbf{i} + y'(t) \mathbf{j} = \lambda ( f_x \mathbf{i} + f_y \mathbf{j} ),  \quad \lambda \not{=} 0 \implies \frac{y'(t)}{x'(t)} = \frac{f_y}{f_x}
					\]
			\item Hence
					\[
						\frac{dy}{dx} = \frac{y'(t)}{x'(t)} = \frac{f_y}{f_x}
					\]
			\item Solve for $y$ in terms of $x$ using the boundary conditions $y=b$ when $x=a$.
		\end{enumerate}

\section{Integration of Functions of Several Real Variables}
	
	\subsection{Multiple integrals}
		\subsubsection{Single integrals}
		
		\subsubsection{Geometric interpretation of single integrals}
		
		\subsubsection{Double integrals}
		\begin{defn}
			The double integral of a continuous function $f(x,y)$ defined on a closed domain $D$ is defined by
			\[
				\iint\limits_D f(x,y) dxdy = \lim_{max|\Delta x_i| \to 0} \lim_{max|\Delta y_j| \to 0} \sum_{j=0}^{m-1} \sum_{i=0}^{n-1} f( \xi_i, \eta_j) \Delta x_i \Delta y_j
			\]
			where
			\[
				\Delta x_i = x_{i+1}- x_i, \quad \Delta y_j = y_{j+1}- y_j, \quad \xi_i \in [x_i, x_{i+1}], \quad \eta_j \in [x_j, x_{j+1}]
			\]
			and the sum for all $(x_i, y_j) \in D$.
		\end{defn}
		
		\subsubsection{Geometric interpretation of double integrals}
		
		\subsubsection{Properties of double integrals}
		\begin{enumerate}
			\item If $D$ is a region of $\mathbb{R}^2$, then $\iint\limits_D f(x,y) dxdy$ is the volume of a solid region $S$ lying above domain $D$ in the $xy$-plane and below the surface $z=f(x,y)$.
			\item $\iint\limits_D 1 dxdy = $ area of $R$.
			\item If $S$ is a region of $\mathbb{R}^3$ then $\iiint\limits_S 1 dxdydz = $ volume of $S$.
			\item Suppose $f$ and $g$ are continuous functions, $c$ and $d$ are constants, then
					\[
						\iint\limits_D [ cf(x,y) + dg(x,y)] dxdy = c \iint\limits_D f(x,y) dxdy + d \iint\limits_D g(x,y) dxdy
					\]
			\item Let $D = D_1 \cup D_2$, where $D_1$ and $D_2$ are non-overlapping regions. If $\iint\limits_D f(x,y)$ dxdy exists, then
					\[
						\iint\limits_D f(x,y) dxdy = \iint\limits_{D_1} f(x,y) dxdy + \iint\limits_{D_2} f(x,y) dxdy
					\]
		\end{enumerate}
		
		\subsubsection{Fubini's theorem}
		\begin{thm}
			Let $f:\mathbb{R}^2 \to \mathbb{R}$ be continuous on $D$, which is given as
			\[
				D = \{(x,y): a \leq x \leq b, c(x) \leq y \leq d(x)\},
			\]
			where $c(x)$ and $d(x)$ are smooth. If $\iint\limits_D f(x,y) dxdy$ exists, then
			\[
				\iint\limits_D f(x,y) dxdy = \int\limits_a^b \left( \int\limits_{c(x)}^{d(x)} f (x,y) dy \right) dx = \int\limits_a^b dx \int\limits_{c(x)}^{d(x)} f (x,y) dy
			\]
		\end{thm}
		
		\begin{thm}
			Let $f:\mathbb{R}^2 \to \mathbb{R}$ be continuous on $D$, which is given as
			\[
				D = \{(x,y): c \leq y \leq d, a(y) \leq y \leq b(y)\},
			\]
			where $a(y)$ and $b(y)$ are smooth. If $\iint\limits_D f(x,y) dxdy$ exists, then
			\[
				\iint\limits_D f(x,y) dxdy = \int\limits_c^d dy \int\limits_{a(y)}^{b(y)} f (x,y) dx
			\]
		\end{thm}
		
		\subsubsection{Fubini's theorem for triple integrals}
		\[
			\iiint\limits_S f(x,y,z) dxdydz \equiv \int\limits_{x=a}^{x=b} dx \int\limits_{y=l(x)}^{y=r(x)} dy \int\limits_{z=g(x,y)}^{z=h(x,y)} f(x,y,z) dz
											\equiv \int\limits_{a}^{b} dx \int\limits_{l(x)}^{r(x)} dy \int\limits_{g(x,y)}^{h(x,y)} f(x,y,z) dz
		\]
		The order of integration does not matter. Choose the easiest order.
		
		\subsubsection{Geometric interpretation of fubini's theorem}
	
	\subsection{Double integrals in polar coordinates}
		\subsubsection{Polar coordinates}
		The relation between polar coordinates and Cartesian coordinates are:
		\[
			x = r \cos{\theta}, y = r \sin{\theta}
		\]
		or
		\[
			r = sqrt{x^2 + y^2}, \tan{\theta} = \frac{y}{x}
		\]
		where $r \geq 0$, $0 \leq \theta \leq 2 \pi$ or $-\pi \le \theta \leq \pi$.
		
		\subsubsection{Double integrals in polar coordinates}
		For double integrals in polar coordinates
		\[
			dA = dxdy = rdrd \theta
		\]
		\[
			\iint\limits_D f(x,y) dxdy = \iint\limits_{D'} f(r,\theta) rdrd \theta
		\]
		where $D'$ is the region of the $(r,\theta)$-plane corresponding to the region of $D$ of the $(x,y)$-plane.
		
		\subsubsection{General case}
		\begin{defn}
			The Jacobian of the functions $x(u,v)$ and $y(u,v)$ with respect to the variables $u$ and $v$ is the determinant. Hence,
			\[
				\frac{\partial (u,v)}{\partial (x,y)} = 
				\begin{vmatrix}
					x_u & x_v  \\
					y_u & y_v 
				\end{vmatrix} 
			\]
		\end{defn}
		For double integrals we have
		\[
			dA = dxdy = \left| \frac{\partial (u,v)}{\partial (x,y)} \right| dudv
		\]
		\[
			\iint\limits_D f(x,y) dxdy = \iint\limits_{D'} f(x,y) | \frac{\partial (x,y)}{\partial (u,v)} | dudv
		\]
		where $D'$ is the region of the $(u,v)$-plane corresponding to the region of $D$ of the $(x,y)$-plane.
	
	\subsection{Change of variables in triple integrals}
		\subsubsection{Cylindrical coordinates}
		Consider the point $P$ with Cartesian coordinates $(x,y,z)$, $Q$ is the projection of $P$ on the $Oxy$ plane. Denote its cylindrical coordinates as $(\rho, \theta, z)$, then:
		\[
			\left\{ 
				\begin{matrix}
					x = r \cos{\theta} \\
					y = r \sin{\theta} \\ 
					z = z
				\end{matrix} 
			\right.
				\iff
			\left\{ 
				\begin{matrix}
					r = \sqrt{x^2 + y^2}  \\
					\tan{\theta} = \frac{y}{x} \\
					z = z
				\end{matrix} 
			\right.
		\]
		where $r \ge 0$, $0 \le \theta < 2 \pi$, $-\infty < z < \infty$.
		
		\subsubsection{Jacobian for triple integrals}
		\begin{defn}
			The Jacobian of the functions $x(u,v,w)$, $y(u,v,w)$, and $z(u,v,w)$ with respect to the variables $u$, $v$, and $w$ is the determinant. Hence,
			\[
				\frac{\partial (u,v,w)}{\partial (x,y,z)} = 
				\begin{vmatrix}
					x_u & x_v & x_w \\
					y_u & y_v & y_w \\
					z_u & z_v & z_w					
				\end{vmatrix} 
			\]
		\end{defn}
		
		For triple integrals we have that 
		\[
			dV = dxdydz = \left| \frac{\partial (x,y,z)}{\partial (u,v,w)} \right| dudvdw
		\]
		Hence,
		\[
			\iiint\limits_S f(x,y,z) dxdydz = \iiint\limits_{S'} f(x(u,v,w), y(u,v,w), z(u,v,w)) \left| \frac{\partial (x,y,z)}{\partial (u,v,w)} \right| dudvdw
		\]
		where $S'$ is the region of the $(u,v,w)$-plane corresponding to the region of $S$ of the $(x,y,z)$-plane.
		
		\subsubsection{Triple integrals in cylindrical coordinates}
		For the cylindrical coordinates we have that $x=r \cos{\theta}$, $y=r \sin{\theta}$, and $z=z$.
		Hence,
		\[
			\frac{\partial (x,y,z)}{\partial (r,\theta,z)} = 
				\begin{vmatrix}
					x_r & x_\theta & x_z \\
					y_r & y_\theta & y_z \\
					z_r & z_\theta & z_z					
				\end{vmatrix}
				=
				\begin{vmatrix}
					x_r & x_\theta & 0 \\
					y_r & y_\theta & 0 \\
					0 & 0 & 1					
				\end{vmatrix} = r
		\]
		Finally giving that
		\[	
			dV = dxdydz = rdrd\theta dz
		\]
		Hence,
		\[
			\iiint\limits_S f(x,y,z) dxdydz = \iiint\limits_{S'} f(r, \theta, z) r drd\theta dz
		\]
		
		\subsubsection{Spherical coordinates}
		Spherical coordinates $(\rho, \theta, \phi)$
		\[
			x = \rho \sin{\phi} \cos{\theta}, \quad y = \rho \sin{\phi} \sin{\theta}, \quad z = \rho \cos{\phi}
		\]
		where $\rho \ge 0$, $0 \le \theta \le 2 \Pi$, $0 \le \phi \le \Pi$.
		
		Relation between Cartesian and spherical coordinates
		\begin{align*}
			\rho &= \sqrt{x^2 + y^2 + z^2} \\
			\theta &= \arctan{ \left( \frac{y}{z} \right)} \\
			\phi &= \arccos{ \left(\frac{z}{\rho} \right)}
		\end{align*}
		
		Relation between cylindrical and spherical coordinates
		\begin{align*}
			r &= \rho \sin{\phi} \\ 
			\theta &= \theta \\
			z &= \rho \cos{\phi}
		\end{align*}
		
		\subsubsection{Jacobian of spherical coordinates}
		Using
		\[
			x = \rho \sin{\phi} \cos{\theta}, \quad y = \rho \sin{\phi} \sin{\theta}, \quad z = \rho \cos{\phi}
		\]
		we have that
		\[
			\frac{\partial (x,y,z)}{\partial (\rho,\phi,\theta)} = \rho^2 \sin{\theta}
		\]
		
		\subsubsection{Triple integrals in spherical coordinates}
		From the above result, we have that
		\[
			dV = \rho^2 \sin{\theta} d\rho d\phi d\theta
		\]
		Hence
		\[
			\iiint\limits_S f(x,y,z) dxdydz = \iiint\limits_{S'} f( \rho, \phi, \theta) \rho^2 \sin{\theta} d\rho d\phi d\theta
		\]
		
	\subsection{Line integrals}
		\subsubsection{Fields}
		\begin{defn}
			A scalar function depending on a position is called a scalar field. A vector function depending on a position is called a vector field.
		\end{defn}
		
		\subsubsection{Tangential vector}
		If $\mathbf{r} = (x(t), y(t), z(t)), \quad t \in [t_0, t_1]$ is a parameterisatino of the curve $C$ (preserving orientation) then
		\begin{itemize}
			\item The tangential vector of the curve is $\mathbf{T}(t) = \frac{d\mathbf{r}}{dt} = \left( \frac{dx}{dt}, \frac{dy}{dt}, \frac{dz}{dt} \right)$
			\item It's magnitude is $\left| \mathbf{T} \right| = \left| \frac{d\mathbf{r}}{dt} \right| = \sqrt{ \left( \frac{dx}{dt} \right)^2 + \left( \frac{dy}{dt} \right)^2 + \left( \frac{dz}{dt} \right)^2   }$
			\item The length $ds$ is $ds = \left| d\mathbf{r} \right| = \left| \frac{d\mathbf{r}}{dt} \right|dt = \left| \mathbf{T} \right| dt$
		\end{itemize}
		
		\subsubsection{Line integrals}
		\begin{defn}
			The line integral of a function $f(\mathbf{r})$ over a curve $C$
			\[
				\mathbf{r} = \mathbf{r}(t) = x(t)\mathbf{i} + y(t)\mathbf{j} + z(t)\mathbf{k}, \quad a \le t \le b
			\]
			is denoted as $\int\limits_C f(\mathbf{r}) ds$, and is defined by
			\[
				\int\limits_C f(\mathbf{r}) ds = \int\limits_a^b f(\mathbf{r}(t))|\mathbf{T}|dt
			\]
		\end{defn}
		
		Some important notes
		\begin{itemize}	
			\item $f(\mathbf{r}(t))$ is $f(\mathbf{r})$ along the curve $C$
			\item $|\mathbf{T}| = \left| \frac{d\mathbf{r}}{dt} \right| = \sqrt{ \left( \frac{dx}{dt} \right)^2 + \left( \frac{dy}{dt} \right)^2 + \left( \frac{dz}{dt} \right)^2 }$
			\item The right hand size is just a common 1-dimensional definite integrate to $t$.
		\end{itemize}
		
		\subsubsection{Line integrals of vector fields}
		\begin{defn}
			The line integral of a vector $v(\mathbf{r})$ over a curve $C$
			\[
				\mathbf{r} = \mathbf{r}(t) = x(t)\mathbf{i} + y(t)\mathbf{j} + z(t)\mathbf{k}, \quad a \le t \le b
			\]
			is denoted as $\int\limits_C \mathbf{v} \cdot d\mathbf{r}$, and is defined by
			\[
				\int\limits_C \mathbf{v} \cdot d\mathbf{r} = \int\limits_a^b \mathbf{v}(\mathbf{r}(t)) \cdot \mathbf{T}(t) dt
			\]
		\end{defn}
		
		Some important notes
		\begin{itemize}
			\item $\mathbf{v}(\mathbf{r}(t))$ is a vector $\mathbf{v}(\mathbf{r})$ along the curve $C$.
			\item TODO
			\item The integrand on the right is a scalar function of $t$
			\item The right hand size is just a common 1-dimensional definite integrate to $t$
		\end{itemize}
		
		\subsubsection{Other expressions of line integrals}
		\subsubsection{Work done by a force}
	

\end{document}


























