\documentclass[12pt]{article}
\usepackage{amsmath}
\usepackage{amsthm}

% Font stuff
\usepackage[T1]{fontenc}
\usepackage[sfmath]{kpfonts}
\renewcommand*\familydefault{\sfdefault}

\usepackage{fullpage}

\newtheorem*{defn}{Definition}
\newtheorem*{lemma}{Lemma}
\newtheorem*{prop}{Proposition}
\newtheorem*{crl}{Corollary}
\newtheorem*{thm}{Theorem}

\date{\today}

\input{./include/keywords.tex}

% SET THIS STUFF - info for front page.
\newcommand{\nameOfTheModule}{MSM2Aa}
\newcommand{\nameOfTheAuthor}{Will Ridgers}
\newcommand{\nameOfTheLecturer}{Dr Qianxi Wang}

\begin{document}
\newcommand{\HRule}{\rule{\linewidth}{0.5mm}}

\begin{titlepage}
\begin{center}


% Upper part of the page
\textsc{\LARGE University of Birmingham}\\[1.5cm]

\textsc{\Large Summary of lectures}\\[0.5cm]


% Title
\HRule \\[0.4cm]
{ \huge \bfseries \nameOfTheModule}\\[0.4cm]
\HRule \\[1.5cm]

% Author
\nameOfTheAuthor

\vfill

% Bottom of the page
{\large \today}

\end{center}
\end{titlepage}
\tableofcontents
\pagebreak

\section{Functions of Several Real Variables}
	\begin{defn} 
		For a function $z = f(x,y)$ a vertical section if the graph of $z = f(x,c)$ or $z = f(c,y)$ 
		for some constant $c$. A level curve is the curve $c = f(x,y)$.
	\end{defn}

\section{Partial derivatives and chain rule}

	\subsection{Partial derivatives}
	\begin{defn}
		The function $f: \mathbb{R}^3 \to \mathbb{R}$ is said to have a partial derivative with respect to $x$
		at the point $(x_0, y_0, z_0)$ if there exists a limit
			\[
				\frac{\partial f (x_0, y_0, z_0)}{\partial x} \equiv \lim_{\Delta x \to 0} 
				\frac{f(x_0 + \Delta x, y_0, z_0) - f(x_0, y_0, z_0)}{\Delta x}
			\]
		which is called the partial derivative of $f$ with respect to $x$.
	\end{defn}
	
	Derivatives with respect to $y$ and $z$ are defined in a similar way.

	\subsection{Differentiable functions}
	\begin{defn}
		The function $f(x,y,z)$ is called differentiable at $(x_0, y_0, z_0)$ if 
		$\Delta f = f(x,y,z) - f(x_0, y_0, z_0)$ can be expressed as
		\[
			\Delta f = f_x(x_0, y_0, z_0) \Delta x + f_y(x_0, y_0, z_0) \Delta y + f_z(x_0, y_0, z_0) \Delta z + o(\rho)
		\]
		where $\Delta x = x - x_0$, $\Delta y = y - y_0$, $\Delta z = z - z_0$ and 
		\[
			\rho = \sqrt{ (\Delta x)^2 + (\Delta y)^2 + (\Delta z)^2 }
		\]
	\end{defn}
	The notation $o(\rho)$ means going to $0$ faster than $\rho$. Note, this is infinitely small so can be ignored.
	
	\subsection{Sufficiently smooth}
	\begin{defn}
		(Informal) A function is said to be sufficiently smooth if the function and as many of it's partial derivatives as required are continuous where they need to be.
	\end{defn}
	
	\subsection{Chain rule}
	\begin{defn}
		Chain Rule for two variables. Let $f$ be a function of the variables $x_1, x_2, \cdots , x_n$, $f = f(x_1, x_2, \cdots , x_n)$ 
		where each $x_j$ is a function of the variables $t_1, t_2, \cdots , t_n$, $x_j = x_j(t_1, t_2, \cdots , t_n)$. Provided
		that $f$ and $x_j$ are sufficiently smooth, then
		\[
			\frac{\partial f}{\partial t_i} = \frac{\partial f}{\partial x_1}\frac{\partial x_1}{\partial t_i} + 
											\frac{\partial f}{\partial x_2}\frac{\partial x_2}{\partial t_i} + 
											\cdots +
											\frac{\partial f}{\partial x_n}\frac{\partial x_n}{\partial t_i}
		\]
	\end{defn}


\section{Vector and vector geometry}

	\subsection{Vectors}
	\begin{defn}
		A vector has both a magnitude and a direction. Vectors may be denoted as $\mathbf{u}$, $\vec{u}$, or $\underline{u}$.
		The magnitude of a vector is denoted by 
		\[
			|\mathbf{u}| = \sqrt{u_1^2 + u_2^2 + u_3^2}
		\]
		A unit vector has length $1$ and in the Cartesian coordinate system $\mathbf{i}$, $\mathbf{j}$, and $\mathbf{k}$ denote the the unit 
		coordinate vectors along x, y, z-axes respectively.
	\end{defn}
	
	\subsection{Scalar (dot) product}
	\begin{defn}
		Given two vectors $\mathbf{a} = a_1 \mathbf{i} + a_2 \mathbf{j} + a_3 \mathbf{k}$,
		$\mathbf{b} = b_1 \mathbf{i} + b_2 \mathbf{j} + b_3 \mathbf{k}$ the scalar (or dot) product of the two is
		\[
			\mathbf{a} \cdot \mathbf{b} = |\mathbf{a}||\mathbf{b}|\cos{\theta} = a_1 b_1 + a_2 b_2 + a_3 b_3
		\]
		where $\theta$ is the angle between $\mathbf{a}$ and $\mathbf{b}$.
	\end{defn}
	
	\subsection{Vector (cross) product}
	\begin{defn}
		Given two vectors $\mathbf{a} = a_1 \mathbf{i} + a_2 \mathbf{j} + a_3 \mathbf{k}$,
		$\mathbf{b} = b_1 \mathbf{i} + b_2 \mathbf{j} + b_3 \mathbf{k}$ the vector (or cross) product of the two is
		\[
			\mathbf{a} \times \mathbf{b} = |\mathbf{a}||\mathbf{b}|\sin{\theta}\mathbf{n} =
			\begin{vmatrix}
				\mathbf{i} & \mathbf{j} & \mathbf{k} \\
				a_1 & a_2 & a_3 \\
				b_1 & b_2 & b_3 
			\end{vmatrix} 
		\]
	\end{defn}
	
	\subsection{Straight line equation}
	\begin{defn}
		The vector equation of a straight line is
		\[
			\mathbf{r} = \mathbf{a} + t \mathbf{d}
		\]
		for $t \in \mathbb{R}$, where $\mathbf{a}$ is a point on the line, $\mathbf{d}$ is parallel to the line, 
		and $\mathbf{r}$ is the position vector of a general point on the line.
	\end{defn}
	
	\subsection{3 dimensional plane equation}
	\begin{defn}
		The vector equation of a plane in 3 dimensions is
		\[
			\mathbf{r} \cdot \mathbf{n} = \mathbf{a} \cdot \mathbf{n}
		\]
		where $\mathbf{a}$ is the position vector of a point in the plane, $\mathbf{n}$ is a vector normal to
		the plane and $\mathbf{r}$ is the position vector of a general point in the plane.
	\end{defn}
	
	\subsection{Parametric form}
	A curve $C$ can be expressed using parametric form
	\[
		\mathbf{r}(t) = x(t)\mathbf{i} + y(t)\mathbf{j} + z(t)\mathbf{k}
	\]
	where $a \leq t \leq b$. Thus the initial point $A$ is $(x(a),y(a),z(a))$ and the terminal point $B$ at $(x(b),y(b),z(b))$.
	$\mathbf{r}(t)$ is a mapping on the interval $[a,b]$ on the t-axis to the curbe in xyz-space.
	
	\subsection{Tangent vector of a curve}

\section{Gradient vector and directional derivative}

\end{document}